%% LyX 2.0.7 created this file.  For more info, see http://www.lyx.org/.
%% Do not edit unless you really know what you are doing.
\documentclass[oribibl]{llncs}
%\usepackage[utf8]{luainputenc}
%\setcounter{secnumdepth}{3}
%\setcounter{tocdepth}{3}
\exhyphenpenalty=10000
\hyphenpenalty=10000
\usepackage{hyperref}
\usepackage{float}
\usepackage{longtable}

%\usepackage[unicode=true,pdfusetitle,
 %bookmarks=true,bookmarksnumbered=false,bookmarksopen=false,
 %breaklinks=false,pdfborder={0 0 1},backref=section,colorlinks=false]
 %{hyperref}

%\makeatletter
%
%%%%%%%%%%%%%%%%%%%%%%%%%%%%%%% LyX specific LaTeX commands.
%%% Because html converters don't know tabularnewline
%\providecommand{\tabularnewline}{\\}
%
%\makeatother

\begin{document}

\title{A Multilevel Classification Method for Classifying Web Pages}
\author{Bharadwaj J(CS13M058) \and \\ Pankaj Kashyap(CS13M035)}
     
\institute{Dept. of CSE, IIT Madras}
\maketitle
\begin{abstract}
Given, a large set of web pages in the retail domain, our aim is to
classify them into Product Pages, Product Listing Pages and Irrelevant
Pages. We had earlier proposed a multilevel classification scheme
for the same, where we attempt to classify based on the less informative
parts of the web page, and if unsuccessful, start looking at the page
content. The scope of this report, is restricted to the latter, where
we classify based on the web page content only. Also, we focus on
the elimination of irrelevant pages, by constructing one-class Support
Vector Machine classifiers on the Product page and Listing page data.
This can be viewed as a \textit{Novelty Detection Problem}, where
the irrelevant pages do not have any model for themselves, but are
treated as outliers or abnormalities.
\end{abstract}

\section{Introduction}

One Class Support Vector Machines was first proposed by Sch\"{o}lkopf
et al \cite{a}. This method constructs a function, that is positive
for a small subset of the data and negative everywhere else. This
can be visualized, as a small ball or hypersphere around the positive
class, separating them from the negative class. This is particularly
useful when the negative class is extremely large, compared to the
positive data or when it is not possible to generate sufficient data
for one of the two classes to train a binary classifier. In our problem,
the irrelevant pages are the negative pages. They include all web pages
that are not product or product listing pages, which is massive compared
to the positive class. Training a model, that captures features of
all such pages is almost impossible. Hence, they are treated as outliers
and the problem itself is reduced to a Novelty Detection Problem. 

The data used for training consisted of 2875 web pages, out of which
1433 were product pages and the remaining 1442 were product listing
pages. These were obtained from around 300 different retailers. Also
for testing, 50 different irrelevant pages were obtained from popular
retail websites like flipkart, amazon and ebay. These included pages
such as user information pages, career pages, shipping rates, policies,
press releases etc. Natural Language Toolkit was used for extracting
the features and Scikit Learn package, based on LIBSVM \cite{c} was
used for training the one-class SVM on the product and product listing
pages.


\section{Methodology}

The entire web page classification process can be divided into the
following four stages: \emph{Document-Vector matrix creation, feature
extraction, training classifiers on the reduced data, testing on new
data.} 


\subsection{Document-Vector model}

Document-Vector model is used to represent the presence or frequency
of words in a set of documents. It is called bag-of-words representation,
where the rows of a matrix represent a document, and the columns represent
the words in the document. The cell values can be either 0/1 indicating
absence or presence of words in documents or they can also hold the
frequency of words in documents. In this experiment, we use the 0/1
based bag of words model, where each row represents a unique web page
and the columns represent the words in the web pages. Thus, we generate
two sets of document-vector matrices: one for the product pages and
one for product-listing pages. 


\subsection{Feature Extraction}

For a html document features can be of two types: \textit{Text Features,}
captured by the bag of words model, and \textit{Structural Features}.
The structural features themselves, might not carry much information
about the web page. However, when combined with the text features,they
are expected to improve the results. At present, the focus is on text
features only. However, in future, ways to combine both these sets
of features and train the model on the combined feature set will be
explored.

The html document may contain stop words, punctuation marks, digits
and empty spaces, which do not contribute to the classification. Hence,
it is necessary to remove features corresponding to these, while vectorizing
the document. 


\subsection{Training and Testing}

The first matrix will have 1483 rows, corresponding to 1433 product
pages and 50 test pages and the second matrix will contain 1492 rows,
corresponding to 1442 listing pages and the same 50 web pages. For
training the product pages model, we took out 133 product pages and
used them as test data along with the generated test pages. Similarly,
for the listing pages 150 were taken out for testing. It is necessary
to shuffle the pages, before taking out the test pages, so as to have
a mix of all retailers in the training data.

One class SVMs using Radial Basis Function kernel \cite{b}, were trained
on these two datasets. Model parameters $\nu$ and $\gamma$ were
varied and the models generated were tried on the test data taken
out. Since, we need to minimize both false positives and false negatives,
the model with the highest f-measure was considered as the best model
and its results are shown below.


\section{Experiments and Results}
The results indicate that, the classifier developed with the listing pages as the positive class performs better that with the product pages as the positive class.

\begin{table}[H]
\setlength{\tabcolsep}{5pt}
\begin{centering}
\caption{Predictions on the test data}

\par\end{centering}

\centering{}%
\begin{tabular}{|c|c|c|c|c|c|c|}
\hline 
\textbf{Positive class} & \textbf{Total Positives} & \textbf{Total Negatives} & \textbf{TP} & \textbf{TN } & \textbf{FP} & \textbf{FN}\tabularnewline
\hline 
\textbf{Product Page} & 133 & 50 & 115 & 32 & 18 & 18\tabularnewline
\hline 
\textbf{Listing Page} & 150 & 50 & 131 & 41 & 1 & 19\tabularnewline
\hline 
\end{tabular}
\end{table}


\begin{table}[H]
\setlength{\tabcolsep}{5pt}

\begin{centering}
\caption{Precision, Recall and F1-Score measures on the test data}

\par\end{centering}

\centering{}%
\begin{tabular}{|c|c|c|c|}
\hline 
\textbf{Positive class} & \textbf{Precision} & \textbf{Recall } & \textbf{f-Measure}\tabularnewline
\hline 
\textbf{Product Page} & 0.865 & 0.865 & 0.865\tabularnewline
\hline 
\textbf{Listing Page} & 0.873 & 0.992 & 0.929\tabularnewline
\hline 
\end{tabular}
\end{table}



\section{Conclusions and Future Work}

The classifiers developed are adequate for eliminating irrelevant
web pages. However, the biggest challenge is develop a classifier that
will differentiate between product web pages and product-listing web pages.
There is a large overlap between the two classes, as the product listing
pages themselves may contain descriptions of a number of products.
Hence, content specific features alone are not sufficient to differentiate
between the two classes. We need to extract features that carry information
about the structure of the web pages. One such approach would be, to
use the html tag distribution by calculating the tag to text ratio
in web pages.

In addition, the multilevel scheme of classification also needs to
be implemented. Along with a series of classifiers at each level,
we also need to develop a confidence measure that indicates the efficiency
of classification at each level. On close examination of the web page
URLs, one can be assured that this will produce good results for a
good percentage of product and listing pages, as the URLs themselves
carry text with sufficient discriminating power. For example, many
of the product page URLs will have the text 'productId=' in the URL.
Similarly, many product listing pages have 'categoryId' in their URLs.
The confidence measure should be sufficiently high for such classifications.
\begin{thebibliography}{[MT1]}
\bibitem[1]{a} B. Sch\"{o}lkopf, R.C. Williamson, A.J. Smola, J.Shawe-Taylor
and J.C. Platt. Support Vector Method for Novelty Detection . Technical
report, Microsoft Research, MSR-TR-99-87, 1999.

\bibitem[2]{b} L. M. Manevitz and M. Yousef. One-class SVMs for document
classification. Journal of Machine Learning Research,2:139-154, 2001

\bibitem[3]{c} C.C. Chang and C. J. Lin. LIBSVM: A Library for Support
Vector Machines (2013) \end{thebibliography}

\end{document}
